% Versionstabelle.

\section{Revisionshistorie}\label{sec:revisionshistorie}
\begin{versionhistory}
	\vhEntry{1.00}{29.04.2024}{Artur Konkel}{Initiales Aufsetzen des Projekts}
	\vhEntry{1.01}{30.04.2024}{Anna-Livia Martin}{Initiales Aufsetzen des Dokuments}
	\vhEntry{1.02}{30.04.2024}{Anna-Livia Martin}{Hinzufügen der allgemeinen Tex-Dateien}
	\vhEntry{1.03}{30.04.2024}{Anna-Livia Martin}{Einfügen von Szenario 2 und zwei Anwendungsfällen}
	\vhEntry{1.04}{01.05.2024}{Artur Konkel}{Einfügen von Szenario 1 "Lagereinrichtung"}
	\vhEntry{1.05}{01.05.2024}{Artur Konkel}{Einfügen des Anwendungsfalls "Neues Lager einrichten"}
	\vhEntry{1.06}{01.05.2024}{Artur Konkel}{Einfügen des Anwendungsfalls "Bestehendes Lager umbauen"}
	\vhEntry{1.07}{02.05.2024}{Vivien Weber}{Einfügen von Szenario 4 "Warenentnahme"}
	\vhEntry{1.08}{06.05.2024}{Artur Konkel}{Hinzufügen von tex.pdf}
	\vhEntry{1.09}{06.05.2024}{Vivien Weber}{Einfügen des Anwendungsfalls 1 "Zutatenpakete aus dem Lager entnehmen für den geschätzten Tagesbedarf" für Szenario 4 - Warenentnahme}
	\vhEntry{1.10}{06.05.2024}{Vivien Weber}{Einfügen des Anwendungsfalls 2 "Dringende Entnahme von Zutaten bei unerwartetem erhöhten Bestellaufkommen" für Szenario 4 - Warenentnahme}
	\vhEntry{1.11}{06.05.2024}{Artur Konkel}{Refactoring der Tex-Datei}
	\vhEntry{1.12}{06.05.2024}{Vivien Weber}{Refactoring der Tex-Datei}
	\vhEntry{1.13}{06.05.2024}{Artur Konkel}{Refactoring der Tex-Datei}
	\vhEntry{1.14}{08.05.2024}{Artur Konkel}{Anpassung des Textes an den Editor}
	\vhEntry{1.15}{08.05.2024}{Anna-Livia Martin}{Finale Überprüfung von Szenario 2 in Bezug auf Wireframes}
	\vhEntry{1.16}{08.05.2024}{Vivien Weber}{Refactoring der Tex-Datei und Anpassung der Texte an die Wireframes}
	\vhEntry{1.17}{08.05.2024}{Anna-Livia Martin}{Anpassung der Texte aufgrund des Fehlens der Funktion "automatisch platzieren"}
	\vhEntry{1.18}{08.05.2024}{Artur Konkel}{Einfügen des UseCase-Diagramms}
	\vhEntry{1.19}{08.05.2024}{Artur Konkel}{Einfügen des UseCase-Diagramm-Bildes}
	\vhEntry{1.20}{08.05.2024}{Anna-Livia Martin}{Hinzufügen von nichtfunktionalen Anforderungen und Formatierung von Texten}
	\vhEntry{1.21}{13.05.2024}{Vivien Weber}{Hinzufügen des Textes für "Zielstellung"}
	\vhEntry{1.22}{13.05.2024}{Anna-Livia Martin}{Hinzufügen von domänenmodell.png}
	\vhEntry{1.23}{13.05.2024}{Anna-Livia Martin}{Hinzufügen der Beschreibung für das Domänenmodell}
	\vhEntry{1.24}{13.05.2024}{Vivien Weber}{Refactoring von Themen und Überschriften}
	\vhEntry{1.25}{13.05.2024}{Artur Konkel}{Austausch der Anwendungsfälle mit Usecase "Regalboden platzieren, Regalwand löschen"}
	\vhEntry{1.26}{14.05.2024}{Vivien Weber}{Hinzufügen der Referenz für Use-Case-Beschreibungen}
	\vhEntry{1.27}{15.05.2024}{Anna-Livia Martin}{Aktualisierung des Use-Case-Diagramms, Hinzufügen der Use-Case-Beschreibung zum Diagramm, Einfügen der Anwendungsfall-Beschreibung zu Paket erstellen und Paket platzieren, Überprüfung der Anwendungsfall-Beschreibung zu Regalboden platzieren und Regalwand löschen}
	\vhEntry{1.28}{16.05.2024}{Anna-Livia Martin}{Umformulierung zu Pakete platzieren (initial)}
	\vhEntry{1.29}{18.05.2024}{Vivien Weber}{Änderung des Textes für "Anwendungsfall – Zutatenpakete aus dem Lager entnehmen"}
	\vhEntry{1.30}{18.05.2024}{Vivien Weber}{Hinzufügen des Textes "Benutzerführung für die Anwendung"}
	\vhEntry{1.31}{18.05.2024}{Vivien Weber}{Änderung des Anwendungsfalls von "Dringende Entnahme von Zutaten bei unerwartetem erhöhten Bestellaufkommen" zu "Zutatenpaket löschen"}
	\vhEntry{1.32}{19.05.2024}{Vivien Weber}{Hinzufügen der Visionshistorie}
	\vhEntry{1.33}{20.05.2024}{Sarah Schwarzer}{Hinzufügen des Textes "Anwendungsszenario 3 – Umräumen/Umlagern" und die Anwendungsfallbeschreibungen "Paket im Lagersystem verschieben" + "Lager im System umbenennen"}
	\vhEntry{1.34}{20.05.2024}{Sarah Schwarzer}{Änderungen in Anwendungsfall "Lager im System umbenennen"}
	\vhEntry{1.35}{20.05.2024}{Sarah Schwarzer}{Hinzufügen des Styleguides}
	\vhEntry{1.36}{24.05.2024}{Artur Konkel}{Behebung kleineren Fehlern in der Textformatierung}
	\vhEntry{1.37}{24.05.2024}{Artur Konkel}{Hinzufügen von Elementen im Glossar und Aktualisierung der main.pdf}
	\vhEntry{1.38}{24.05.2024}{Vivien Weber}{Hinzufügen von Wireframes_SWT.pdf}
	\vhEntry{1.39}{07.06.2024}{Vivien Weber}{Korrigierte zu Drag \& Drop}
    \vhEntry{1.40}{07.06.2024}{Vivien Weber}{Entfernen von Windows und MacOS bei 2.2.2 Technologsiche Anforderungen}
    \vhEntry{1.41}{07.06.2024}{Vivien Weber}{Hinzufügen der Autoren für die Anwendungsszenarien}
    \vhEntry{1.42}{07.06.2024}{Anna-Livia Martin}{Korrigierung Anna-Livias Texte}
    \vhEntry{1.43}{07.06.2024}{Vivien Weber}{Änderungen Rechtschreibung für Zahlen < 13}
    \vhEntry{1.44}{07.06.2024}{Vivien Weber}{Entfernung des Abschnitts Use-Case-Diagramm mit Beschreibung
    \vhEntry{1.45}{07.06.2024}{Vivien Weber}{Korrigierung des "Anwendungsszenarios 4 - Warenentnahme"}
    \vhEntry{1.46}{07.06.2024}{Vivien Weber}{Korrigierung des Anwendungsfalls "Zutatenpakete aus dem Lager entfernen"}
    \vhEntry{1.47}{07.06.2024}{Vivien Weber}{Korrigierung des Anwendungsfalls "Zutatenpaket löschen"}
    \vhEntry{1.48}{07.06.2024}{Vivien Weber}{Korrigierung Rechtschreibung zu LaTeX}
    \vhEntry{1.49}{07.06.2024}{Vivien Weber}{Entfernung der Matrikelnummern}
    \vhEntry{1.50}{07.06.2024}{Artur Konkel}{Aktualisierung Anforderungsspezifikation.pdf und Domänenmodell}
    \vhEntry{1.51}{09.06.2024}{Sarah Schwarzer}{Änderungen an Texten und Hinzufügen der neuen Wireframes}
    \vhEntry{1.52}{10.06.2024}{Anna-Livia Martin}{Erneute Korrigierung Anna-Livias Texte}
    \vhEntry{1.53}{14.06.2024}{Vivien Weber}{Änderungenen Rechtschreibung zu Zutatenpaket}
    \vhEntry{1.54}{14.06.2024}{Anna-Livia Martin}{Korrigierung Spec Dokument mit Kommentaren Weitz}
    \vhEntry{1.54}{15.06.2024}{Sarah Schwarzer}{Änderungen am Styleguide}
    \vhEntry{1.55}{15.06.2024}{Sarah Schwarzer}{Hinzufügen der Bedingungen für die Qualitätsanfoderungen}
    \vhEntry{1.56}{16.06.2024}{Sarah Schwarzer}{Ergänzungen in der Revisionshistorie}
    \vhEntry{1.57}{16.06.2024}{Sarah Schwarzer}{Erneute Änderungen an Texten}
\end{versionhistory}