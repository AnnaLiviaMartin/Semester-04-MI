% Vorlage für Praktikumsberichte
%
% Glossar.tex
%
% Glossar

\section{Ergänzende Bausteine}

\subsection{Glossar}
Das Glossar benutzter Begriffe in Doxis' Umgebung.

\begin{table}[h]
  \centering
  \begin{tabular}{|c|p{0.55\linewidth}|}
  \hline
    \textbf{Java} & \textbf{plattformunabhängige, objektorientierte Programmiersprache, die weit verbreitet für Web-
    , Mobil- und Unternehmensanwendungen verwendet wird} \\
    \textbf{CSS3} & \textbf{Cascading Style Sheets-Standards, der für die Gestaltung und das Layout von Webseiten verwendet wird.} \\
    \textbf{Paket} & \textbf{Darstellung eines Paket mit Maßen und Gewicht} \\
    \textbf{Zutat} & \textbf{Darstellung einer Zutat} \\
    \textbf{Zutatpaket} & \textbf{Kombination aus Zutat und Paket} \\
    \textbf{Regal} & \textbf{Selbst definierter und zusammengesetzer Bereich mit der Möglichkeit Zutaten abzustellen
  .} \\
    \textbf{Drag \& Drop} & \textbf{Interaktionsmöglichkeit mit Objekten, diese per Mausklick zu z iehen und
  woanders zu platzieren} \\
    \textbf{Inhaltsunverträglichkeit} & \textbf{Zutaten können mit anderen Zutaten Unverträglichkeiten bilden und
  können somit nicht zusammen eingelagert werden.} \\
    \textbf{Login} & \textbf{Authentifizierung des Benutzers am System} \\
    \textbf{Warenkorb} & \textbf{Sammlung der Zutaten in einem Bereich zur Entnahme.} \\

  \hline
  \end{tabular}
  \caption{Glossar}
\end{table}